\documentclass[12pt, a4paper]{article}
\usepackage[utf8]{inputenc}
\usepackage[portuguese]{babel}
\usepackage{amsmath, amssymb}
\usepackage{graphicx}
\usepackage{geometry}
\usepackage{float}
\usepackage{listings}
\usepackage{xcolor}
\usepackage{hyperref}

% Configuration for Code Listings
\definecolor{codegreen}{rgb}{0,0.6,0}
\definecolor{codegray}{rgb}{0.5,0.5,0.5}
\definecolor{codepurple}{rgb}{0.58,0,0.82}
\definecolor{backcolour}{rgb}{0.95,0.95,0.92}

\lstdefinestyle{mystyle}{
    backgroundcolor=\color{backcolour},   
    commentstyle=\color{codegreen},
    keywordstyle=\color{magenta},
    numberstyle=\tiny\color{codegray},
    stringstyle=\color{codepurple},
    basicstyle=\ttfamily\footnotesize,
    breakatwhitespace=false,         
    breaklines=true,                 
    captionpos=b,                    
    keepspaces=true,                 
    numbers=left,                    
    numbersep=5pt,                  
    showspaces=false,                
    showstringspaces=false,
    showtabs=false,                  
    tabsize=2,
    literate={á}{{\'a}}1 {à}{{\`a}}1 {ã}{{\~a}}1 {é}{{\'e}}1 {ê}{{\^e}}1 {í}{{\'i}}1 {ó}{{\'o}}1 {õ}{{\~o}}1 {ú}{{\'u}}1 {ü}{{\"u}}1 {ç}{{\c{c}}}1 {Á}{{\'A}}1 {À}{{\`A}}1 {Ã}{{\~A}}1 {É}{{\'E}}1 {Ê}{{\^E}}1 {Í}{{\'I}}1 {Ó}{{\'O}}1 {Õ}{{\~O}}1 {Ú}{{\'U}}1 {Ü}{{\"U}}1 {Ç}{{\c{C}}}1
}

\lstset{style=mystyle}
\geometry{a4paper, margin=2.5cm}

\title{\textbf{Relatório Técnico Detalhado}\\ \large Projeto de Controle para Servomecanismo de Posição}
\author{Gabriel, Felipe, Cintia, Dierson, Guilherme, Nicolas}
\date{\today}

\begin{document}

\maketitle
\tableofcontents
\newpage

\section{Introdução}
Este relatório documenta integralmente o processo de modelagem, análise e projeto de controle para um servomecanismo de posição. O objetivo primordial é garantir precisão e rapidez na resposta do sistema, satisfazendo requisitos estritos de desempenho no domínio do tempo e da frequência.

\section{Modelagem Matemática - Gabriel}

\subsection{Função de Transferência}
O sistema é um servomecanismo controlado por armadura, cuja dinâmica é governada pela interação elétrica e mecânica do motor DC acoplado a uma carga. Os parâmetros identificados são:

\begin{itemize}
    \item Ganho do Motor: $K_m = 1.1$
    \item Polo Mecânico: $a_m = 13.2$ rad/s
    \item Polo Elétrico: $a_e = 950$ rad/s
    \item Ganho do Sistema Acoplado: $K_{sys} = 772$
\end{itemize}

A Função de Transferência de malha aberta $G(s)$ é dada por:

\[
G(s) = \frac{K_m K_{sys}}{s(s+a_m)(s+a_e)}
\]

Substituindo os valores numéricos:

\[
G(s) = \frac{1.1 \times 772}{s(s+13.2)(s+950)} = \frac{849.2}{s(s+13.2)(s+950)}
\]

Expandindo o denominador para a forma polinomial $s^3 + a_2 s^2 + a_1 s + a_0$:

\[
\text{Den}(s) = s(s^2 + (13.2+950)s + (13.2 \times 950)) 
\]
\[
\text{Den}(s) = s(s^2 + 963.2s + 12540) = s^3 + 963.2s^2 + 12540s
\]

Portanto, a função final é:
\begin{equation}
G(s) = \frac{849.2}{s^3 + 963.2s^2 + 12540s}
\end{equation}

\subsection{Análise de Malha Aberta}
\begin{itemize}
    \item \textbf{Polos:} $s_1 = 0$, $s_2 = -13.2$, $s_3 = -950$.
    \item \textbf{Tipo do Sistema:} Tipo 1 (devido ao polo na origem). Isso implica que o erro de regime estacionário para uma entrada degrau é naturalmente nulo.
    \item \textbf{Estabilidade:} O sistema é marginalmente estável em malha aberta devido ao polo na origem.
\end{itemize}

A Figura \ref{fig:pzmap} ilustra a localização dos polos no plano complexo.

\begin{figure}[H]
    \centering
    \includegraphics[width=0.6\textwidth]{../assets/report_images/pzmap_light.png}
    \caption{Mapa de Polos e Zeros do Sistema (Malha Aberta)}
    \label{fig:pzmap}
\end{figure}

A resposta ao degrau em malha aberta (Figura \ref{fig:openloop}) confirma o comportamento integrador (rampa na saída para entrada constante).

\begin{figure}[H]
    \centering
    \includegraphics[width=0.7\textwidth]{../assets/report_images/step_openloop_light.png}
    \caption{Resposta ao Degrau em Malha Aberta (Comportamento Integrador)}
    \label{fig:openloop}
\end{figure}

\section{Controlador Proporcional (P) - Felipe}

\subsection{Especificações e Sintonia}
O objetivo inicial foi sintonizar um ganho $K_p$ que atendesse:
\begin{itemize}
    \item Overshoot ($M_p$): $5\% - 15\%$
    \item Tempo de acomodação ($t_s$): $0.5s - 1.0s$
\end{itemize}

A malha fechada tem equação característica:
\[
1 + K_p G(s) = 0 \implies s^3 + 963.2s^2 + 12540s + 849.2 K_p = 0
\]

Através do Lugar das Raízes (Root Locus), variamos $K_p$. 
Para $K_p = 138$, os polos dominantes de malha fechada se posicionam de forma a fornecer o amortecimento $\zeta$ necessário para o overshoot de 10\%.

\begin{figure}[H]
    \centering
    \includegraphics[width=0.8\textwidth]{../assets/report_images/felipe_rlocus.png}
    \caption{Lugar das Raízes para o Controlador Proporcional}
    \label{fig:rlocus_p}
\end{figure}

\subsection{Desempenho (Simulação)}
Com $K_p = 138$, a simulação (Figura \ref{fig:step_p}) apresentou:
\begin{itemize}
    \item $M_p \approx 10.05\%$
    \item $t_s \approx 0.531s$
\end{itemize}

\begin{figure}[H]
    \centering
    \includegraphics[width=0.8\textwidth]{../assets/report_images/step_response_P.png}
    \caption{Resposta ao Degrau - Controlador P ($K_p=138$)}
    \label{fig:step_p}
\end{figure}

\section{Compensador Lag (Atraso de Fase) - Dierson}

Apesar do bom desempenho transitório do controlador P, o erro de seguimento para entradas em rampa ($r(t) = t$) ainda pode ser melhorado. O compensador Lag visa aumentar o ganho em baixas frequências (Ganho DC) sem alterar significativamente o Lugar das Raízes na região de alta frequência (onde o transiente é definido).

\subsection{Cálculo do Erro em Rampa (Sem Compensação)}
A constante de erro de velocidade $K_v$ para o sistema com controlador P simples ($K_p=138$) seria:
\[
K_v = \lim_{s\to 0} s \cdot K_p G(s) = 138 \times \lim_{s\to 0} \frac{849.2}{(s+13.2)(s+950)}
\]
\[
K_v = 138 \times \frac{849.2}{12540} = 138 \times 0.0677 \approx 9.34 \text{ s}^{-1}
\]
O erro estacionário para rampa unitária é:
\[
e_{ss} = \frac{1}{K_v} = \frac{1}{9.34} \approx 0.107 \quad (10.7\%)
\]
Este valor de 10.7\% é superior ao desejado de 1\%. Precisamos aumentar $K_v$ por um fator de aproximadamente 10.

\subsection{Projeto do Compensador}
O compensador tem a forma:
\[
C_{lag}(s) = K \frac{s + z}{s + p}
\]
Escolhemos a relação $\beta = z/p = 10$ para ganhar uma década em magnitude DC.
Para não afetar a fase na frequência de cruzamento (transiente), escolhemos o polo e o zero muito próximos da orígem.

\textbf{Parâmetros Selecionados:}
\begin{itemize}
    \item Zero: $z = 0.10$
    \item Polo: $p = 0.01$ (uma década abaixo)
    \item Ganho: $K = 150.6$ (Ajuste fino do ganho proporcional para re-sintonizar o overshoot)
\end{itemize}

\subsection{Novo Cálculo de Erro}
\[
K_v^{new} = \lim_{s\to 0} s \cdot C_{lag}(s) G(s) = \lim_{s\to 0} s \cdot \left( 150.6 \frac{s+0.1}{s+0.01} \right) \frac{849.2}{s(s+13.2)(s+950)}
\]
\[
K_v^{new} = 150.6 \cdot \left( \frac{0.1}{0.01} \right) \cdot \frac{849.2}{12540} = 150.6 \cdot 10 \cdot 0.0677 \approx 101.95
\]
\[
e_{ss}^{new} = \frac{1}{101.95} \approx 0.0098 \quad (\mathbf{0.98\%})
\]
O erro caiu para menos de 1\%, cumprindo o requisito.

\subsection{Análise Frequencial e Temporal}
A Figura \ref{fig:bode_compare} mostra o Diagrama de Bode, evidenciando o aumento de ganho em baixas frequências (lado esquerdo) devido ao Lag, enquanto a margem de fase em altas frequências permanece preservada.

\begin{figure}[H]
    \centering
    \includegraphics[width=1.0\textwidth]{../assets/report_images/compare_bode_v2.png}
    \caption{Comparação de Diagramas de Bode (Malha Aberta): Planta Original, com Controlador P, e com Compensador Lag}
    \label{fig:bode_compare}
\end{figure}

A Figura \ref{fig:bode_compare} evidencia como o compensador Lag (curva verde) eleva a magnitude em baixas frequências (lado esquerdo) em comparação ao controlador Proporcional (curva laranja), garantindo maior ganho DC e menor erro estacionário, enquanto mantém a margem de fase e magnitude em altas frequências.

A Figura \ref{fig:rlocus_detail} mostra em detalhe o dipolo polo-zero introduzido próximo à origem.

\begin{figure}[H]
    \centering
    \includegraphics[width=0.7\textwidth]{../assets/report_images/rlocus_lag_detail.png}
    \caption{Detalhe do Lugar das Raízes próximo à origem (Dipolo do Compensador Lag)}
    \label{fig:rlocus_detail}
\end{figure}

Finalmente, apresentamos as comparações diretas de desempenho temporal.

\begin{figure}[H]
    \centering
    \includegraphics[width=1.0\textwidth]{../assets/report_images/compare_step.png}
    \caption{Comparação de Resposta ao Degrau em Malha Fechada}
    \label{fig:step_compare}
\end{figure}

A resposta ao degrau (Figura \ref{fig:step_compare}) confirma que o comportamento transitório do Lag (Verde) é muito próximo ao do Proporcional (Laranja), com um overshoot levemente maior mas ainda dentro das especificações.

\begin{figure}[H]
    \centering
    \includegraphics[width=1.0\textwidth]{../assets/report_images/compare_ramp.png}
    \caption{Comparação de Resposta à Rampa: O Lag praticamente elimina o erro de seguimento visível na curva do Proporcional.}
    \label{fig:ramp_compare}
\end{figure}

\section{Compensador Lead (Avanço de Fase) - Cintia}

Enquanto o controlador Proporcional oferece um bom desempenho, o Compensador Lead (Avanço de Fase) é projetado para modificar o Lugar das Raízes, ``puxando-o'' para a esquerda no plano complexo. Isso permite aumentar a estabilidade relativa e, principalmente, a velocidade de resposta do sistema.

\subsection{Projeto do Compensador}
A função de transferência do compensador Lead é dada por:
\[
C_{lead}(s) = K \frac{s + z}{s + p}, \quad \text{onde } |p| > |z|
\]

A estratégia adotada foi utilizar o zero do compensador para cancelar (ou reduzir significativamente) o efeito do polo dominante da planta mais lento ($s = -13.2$), e posicionar o polo do compensador bem afastado da origem ($s = -100$) para contribuir com ângulo de fase positivo na região de cruzamento de ganho.

\textbf{Parâmetros Selecionados:}
\begin{itemize}
    \item Zero: $z = 20$ (Próximo ao polo de -13.2)
    \item Polo: $p = 100$ (Afastado para extender a largura de banda)
    \item Ganho: $K = 700$ (Ajustado para performance máxima sem saturação excessiva)
\end{itemize}

\subsection{Análise no Lugar das Raízes e Bode}
A Figura \ref{fig:rlocus_lead} mostra como o compensador altera a trajetória dos polos de malha fechada. Note como os ramos se curvam mais profundamente para o semi-plano esquerdo, permitindo respostas mais rápidas.

\begin{figure}[H]
    \centering
    \includegraphics[width=0.8\textwidth]{../assets/report_images/root_locus_Lead.png}
    \caption{Lugar das Raízes com Compensador Lead}
    \label{fig:rlocus_lead}
\end{figure}

A Figura \ref{fig:rlocus_lead_detail} abaixo detalha o efeito do cancelamento. O zero em -20 atrai o polo da planta em -13.2, reduzindo drasticamente sua influência na resposta temporal.

\begin{figure}[H]
    \centering
    \includegraphics[width=0.8\textwidth]{../assets/report_images/rlocus_lead_detail.png}
    \caption{Detalhe do Cancelamento Polo-Zero: O Zero do compensador ``anula'' o Polo lento da planta}
    \label{fig:rlocus_lead_detail}
\end{figure}

O diagrama de Bode (Figura \ref{fig:bode_lead}) confirma o aumento da largura de banda e a injeção de fase na região de média frequência.

\begin{figure}[H]
    \centering
    \includegraphics[width=0.8\textwidth]{../assets/report_images/bode_Lead.png}
    \caption{Diagrama de Bode do Sistema Compensado (Lead)}
    \label{fig:bode_lead}
\end{figure}

\subsection{Desempenho Temporal}
O resultado no domínio do tempo foi extremamente positivo. O sistema tornou-se muito mais ágil.

\begin{figure}[H]
    \centering
    \includegraphics[width=0.8\textwidth]{../assets/report_images/step_response_Lead.png}
    \caption{Resposta ao Degrau com Compensador Lead ($t_s \approx 0.28s$)}
    \label{fig:step_lead}
\end{figure}

Comparado ao controlador Proporcional, o Tempo de Acomodação ($t_s$) caiu de $\approx 0.53s$ para $\mathbf{0.28s}$, uma redução de quase 50\%, com um overshoot mínimo, demonstrando a eficácia da estratégia de avanço de fase para regimes transitórios exigentes.

\section{Controlador Lead-Lag Integrado (Solução Combinada) - Conjunto}

Para obter o ``melhor dos dois mundos'' — a precisão em regime permanente do Lag e a velocidade de resposta do Lead — projetamos um controlador Lead-Lag em cascata. Essa abordagem visa satisfazer simultaneamente todos os requisitos de desempenho de forma robusta.

\subsection{Estratégia de Projeto Detalhada}

A concepção deste controlador baseou-se em atacar os dois problemas fundamentais do sistema de forma desacoplada:

\subsubsection{1. O ``Acelerador'' (Compensador Lead)}
O sistema original possui um polo dominante em $s = -13.2$, que limita severamente a velocidade de resposta.
\begin{itemize}
    \item \textbf{Zero ($z_{lead} = 20$):} Escolhido estrategicamente para \textbf{cancelar} (ou dominar) o efeito do polo lento da planta. Ao posicionar um zero próximo ao polo dominante, anulamos sua influência retardadora no Lugar das Raízes.
    \item \textbf{Polo ($p_{lead} = 100$):} Posicionado distante da origem para fornecer um amplo avanço de fase na região de frequências médias e altas, permitindo aumentar a largura de banda sem comprometer a estabilidade.
\end{itemize}

\subsubsection{2. O ``Corretor'' (Compensador Lag)}
Para garantir precisão, precisávamos aumentar o ganho em baixa frequência sem prejudicar o transiente rápido obtido com o Lead.
\begin{itemize}
    \item \textbf{Dipolo ($z_{lag}=0.1, p_{lag}=0.01$):} Posicionado muito próximo à origem para não alterar o Lugar das Raízes na região transiente.
    \item \textbf{Relação $\beta = 10$:} A escolha da razão $z/p = 10$ multiplica o ganho DC da malha por 10. Isso reduz o erro estacionário em uma ordem de grandeza, garantindo erro zero para degrau e baixíssimo para rampa.
\end{itemize}

\subsubsection{3. Sintonia Fina do Ganho ($K$)}
Com o Lead garantindo margem de fase e o Lag garantindo ganho DC, pudemos ser mais agressivos no ganho proporcional.
\begin{itemize}
    \item \textbf{Ganho $K=1000$:} Este valor elevado foi possibilitado pela estabilização dinâmica do Lead. Ele minimiza ainda mais os erros e acelera a resposta, resultando em um tempo de acomodação de apenas $0.37s$.
\end{itemize}

A Função de Transferência final resultante é:
\[
C(s) = 1000 \cdot \underbrace{\left( \frac{s+0.1}{s+0.01} \right)}_{\text{Lag (Precisão)}} \cdot \underbrace{\left( \frac{s+20}{s+100} \right)}_{\text{Lead (Velocidade)}}
\]

\subsection{Resultados Finais}
A resposta ao degrau (Figura \ref{fig:step_leadlag}) demonstra um desempenho excepcional, superior a qualquer controlador isolado.

\begin{itemize}
    \item \textbf{Overshoot ($M_p$):} $\mathbf{3.19\%}$ (Excelente, muito abaixo do limite de 15\%)
    \item \textbf{Tempo de Acomodação ($t_s$):} $\mathbf{0.37s}$ (Muito rápido, bem abaixo de 1.0s)
    \item \textbf{Erro Estacionário:} Virtualmente zero (devido à ação integral do Lag).
\end{itemize}

\begin{figure}[H]
    \centering
    \includegraphics[width=0.9\textwidth]{../assets/report_images/step_response_LeadLag.png}
    \caption{Resposta Final do Sistema com Controlador Lead-Lag Integrado}
    \label{fig:step_leadlag}
\end{figure}

O Diagrama de Bode da malha combinada (Figura \ref{fig:bode_leadlag}) mostra a modelagem da resposta em frequência em toda a faixa de operação.

\begin{figure}[H]
    \centering
    \includegraphics[width=0.9\textwidth]{../assets/report_images/bode_LeadLag.png}
    \caption{Diagrama de Bode Final (Lead-Lag)}
    \label{fig:bode_leadlag}
\end{figure}

\section{Controlador PID (Proporcional-Integral-Derivativo) - Guilherme}

Além das estratégias clássicas de compensação em frequência (Lead/Lag), implementamos um controlador PID sintonizado via método de Ziegler-Nichols e refinado empiricamente.

\subsection{Sintonia e Estrutura}
A estrutura implementada inclui um polo de filtro na ação derivativa para garantir a realizabilidade física (função de transferência própria):
\[
C_{PID}(s) = K_p + \frac{K_i}{s} + \frac{K_d s}{\tau s + 1}
\]
Com $\tau = 0.001$ (filtro rápido).

\textbf{Ganhos Sintonizados:}
\begin{itemize}
    \item $K_p = 200$
    \item $K_i = 100$
    \item $K_d = 5$
\end{itemize}

\subsection{Desempenho}
A Figura \ref{fig:step_pid} apresenta a resposta ao degrau. O controlador PID oferece uma resposta extremamente robusta, eliminando o erro estacionário (ação Integral) e fornecendo amortecimento vigoroso (ação Derivativa) para conter o overshoot causado pelo alto ganho proporcional.

\begin{figure}[H]
    \centering
    \includegraphics[width=0.9\textwidth]{../assets/report_images/step_response_PID.png}
    \caption{Resposta ao Degrau com Controlador PID (Ziegler-Nichols Refinado)}
    \label{fig:step_pid}
\end{figure}

\section{Análise de Robustez (Fatores Paramétricos) - Nicolas}

Para garantir que o controlador projetado funcione adequadamente em condições reais, onde os parâmetros do sistema podem variar (devido a aquecimento, desgaste ou carga variável), realizamos uma análise de robustez baseada em cenários.

\subsection{Cenários de Teste}
Definimos três cenários de operação para o servomecanismo:

\begin{enumerate}
    \item \textbf{Nominal:} Parâmetros ideais de projeto ($K_m = 1.1$, $a_m = 13.2$).
    \item \textbf{Pesado:} Simula um motor enfraquecido e maior atrito viscoso.
    \begin{itemize}
        \item $K_m = 0.8$ (Redução de 27\% no torque)
        \item $a_m = 15.0$ (Maior atrito/amortecimento)
        \item $a_e = 1100$
    \end{itemize}
    \item \textbf{Agressivo:} Simula um motor mais forte e menor atrito.
    \begin{itemize}
        \item $K_m = 1.2$ (Aumento de 9\% no torque)
        \item $a_m = 10.0$ (Menor atrito)
        \item $a_e = 800$
    \end{itemize}
\end{enumerate}

\subsection{Resultados da Análise}
A Figura \ref{fig:robustness} apresenta a resposta ao degrau do controlador Lead-Lag final para os três cenários.

\begin{figure}[H]
    \centering
    \includegraphics[width=0.9\textwidth]{../assets/report_images/robustness_Lead-Lag.png}
    \caption{Robustez do Controlador Lead-Lag: O sistema permanece estável e com desempenho aceitável em todos os cenários.}
    \label{fig:robustness}
\end{figure}

Observa-se que mesmo no cenário ``Agressivo'' (linha tracejada vermelha), onde o ganho de malha aberta é maior e o polo dominante é mais lento, o controlador Lead-Lag mantém o sistema estável com um aumento marginal no overshoot. No cenário ``Pesado'' (linha azul), o sistema é ligeiramente mais lento, mas sem oscilações.

Essa análise confirma que a margem de fase e de ganho projetadas são suficientes para absorver incertezas paramétricas significativas.

\section{Conclusão}
O projeto atingiu todos os objetivos. O modelo foi corretamente identificado e validado. O controlador proporcional $K_p=138$ estabeleceu a base de desempenho dinâmico, o compensador Lag ($K=150.6, z=0.1, p=0.01$) refinou a precisão estacionária reduzindo o erro de rampa para 0.98\%, e o compensador Lead ($K=700, z=20, p=100$) entregou a resposta mais rápida do conjunto ($0.28s$).

\newpage
\appendix
\section{Documentação Complementar de Códigos}
Os códigos desenvolvidos utilizam a biblioteca \texttt{python-control} para modelagem e análise. Abaixo encontra-se o detalhamento funcional de cada script.

\subsection{Modelagem do Sistema (model.py)}
Este script é responsável por definir a estrutura matemática da planta.
\begin{itemize}
    \item \textbf{Função \texttt{define\_system()}:} Constrói as matrizes de espaço de estados (A, B, C, D) usando os parâmetros físicos fornecidos (ganhos $K_m, K_{sys}$ e polos $a_m, a_e$). Retorna o objeto de sistema \texttt{ss}.
    \item \textbf{Configuração Gráfica:} Implementa lógica para alternar entre gráficos para apresentação (escuros/transparentes) e para este relatório (claros/fundo branco), garantindo legibilidade em ambos os meios.
    \item \textbf{Análise:} Gera o mapa de polos e zeros e a resposta ao degrau em malha aberta para validação inicial.
\end{itemize}

\lstinputlisting[language=Python, caption={Script de definição e análise da planta em malha aberta}]{../simulations/model.py}

\subsection{Projeto dos Controladores (controllers.py)}
Este script realiza o design iterativo dos compensadores e gera as validações gráficas.
\begin{itemize}
    \item \textbf{Controlador P:} Simula o efeito do ganho proporcional, gerando o Lugar das Raízes para escolha do ganho ótimo ($K_p=138$) com base no coeficiente de amortecimento.
    \item \textbf{Compensador Lag:} Implementa a lógica de compensação por atraso de fase.
    \begin{itemize}
        \item Define a função de transferência do controlador: $C(s) = K \frac{s+z}{s+p}$.
        \item Calcula, para cada simulação, métricas precisas: Overshoot, Tempo de Acomodação (critério de 2\%) e Erro Estacionário.
        \item Gera gráficos comparativos de Root Locus e Bode para demonstrar a eficácia da compensação na baixa frequência sem alterar a estabilidade transitória.
    \end{itemize}
    \item \textbf{Automação:} O bloco \texttt{\_\_main\_\_} executa as funções de projeto sequencialmente, garantindo que qualquer alteração nos parâmetros seja refletida automaticamente em todos os gráficos de saída.
\end{itemize}

\lstinputlisting[language=Python, caption={Script de projeto e simulação dos controladores P e Lag}]{../simulations/controllers.py}

\end{document}
