\documentclass[12pt]{article}
\usepackage{amsmath, amssymb}
\usepackage{geometry}
\geometry{a4paper, margin=2cm}
\includegraphics\usepackage{graphicx}
\usepackage{listings}
\usepackage{xcolor}

\title{Análise Completa do Ganho DC, Constante de Velocidade e Projeto do Compensador Lag}
\author{}
\date{}

\begin{document}
\maketitle

\section*{1. Modelo da Planta}

A planta fornecida é:
\[
G(s) = \frac{1.2}{s(s+13.2)(s+950)}.
\]

Trata-se de um sistema do tipo 1, pois possui um polo em \(s=0\), logo seu erro de regime permanente para entrada degrau é zero.

\section*{2. Observação Sobre o Integrador}

O integrador presente na planta implica:
\[
G(0) = \infty.
\]

Dessa forma, falar em ``aumentar o ganho DC'' não significa literalmente aumentar \(G(0)\), mas sim aumentar o ganho na região de baixas frequências, ou seja, aumentar a constante de velocidade \(k_v\), que afeta o erro em rampa.

\section*{3. Constante de Velocidade Sem Compensador}

Para malha unitária:
\[
k_v = \lim_{s\to 0} s\,L(s) = \lim_{s\to 0} s\,G(s)C(s).
\]

Sem compensador:
\[
C(s)=1.
\]

Logo:
\[
k_{v0} = \lim_{s\to 0} s \cdot \frac{1.2}{s(s+13.2)(s+950)}
      = \frac{1.2}{13.2 \cdot 950}.
\]

Cálculo:
\[
13.2 \times 950 = 12540,
\]
\[
k_{v0} = \frac{1.2}{12540} = \frac{1}{10450}.
\]

Assim:
\[
k_{v0} \approx 9.5694\times 10^{-5}.
\]

Erro em rampa:
\[
e_{\text{rampa},0} = \frac{1}{k_{v0}} = 10450.
\]

\section*{4. Compensador Lag}

Considere:
\[
C_{\text{lag}}(s)=\frac{s+b}{s+a}, \quad b=10a.
\]

\section*{5. Ganho DC do Compensador}

No ponto \(s=0\):
\[
C_{\text{lag}}(0) = \frac{b}{a} = 10.
\]

Portanto o compensador multiplica a constante \(k_v\) por 10.

\section*{6. Constante de Velocidade com Compensador Lag}

\[
k_v = k_{v0} \cdot \frac{b}{a} 
    = 10\,k_{v0}
    = \frac{10}{10450}
    = \frac{1}{1045}.
\]

Numérico:
\[
k_v \approx 9.5694\times 10^{-4}.
\]

Erro em rampa:
\[
e_{\text{rampa}} = \frac{1}{k_v} = 1045.
\]

\section*{7. Escolha de \(a\)}

Em compensadores lag:
\[
a \approx \frac{\omega_c}{10},
\]
onde \(\omega_c\) é a frequência de cruzamento sem compensador. Quando não se conhece \(\omega_c\), escolhe-se valores como:
\[
a = 0.01,\; 0.05,\; 0.1,
\]
avaliando o impacto no Bode até atingir o afundamento desejado.

\section*{8. Influência no Bode}

O lag:

- reduz o ganho em alta frequência,
- mantém a fase quase inalterada,
- desloca a curva de magnitude para baixo nas baixas frequências,
- melhora \(k_v\) sem afetar muito as margens de estabilidade.

\section*{9. Verificação de Estabilidade}

É necessário garantir:

- margem de fase adequada (típ. > 30°),
- margem de ganho positiva,
- diagrama de Nyquist sem envolver o ponto crítico \(-1+j0\).

\section*{10. Efeito no Dominante do Sistema}

Como o compensador lag adiciona polos e zeros próximos, seu impacto nas raízes dominantes é pequeno. Ele atua mais na parte de baixas frequências.

\section*{11. Resumo Numérico Obtido}

Antes do lag:

\[
k_{v0} = \frac{1}{10450}, \quad e_{\text{rampa},0} = 10450.
\]

Com lag \(b/a=10\):

\[
k_v = \frac{1}{1045}, \quad e_{\text{rampa}} = 1045.
\]

Redução de 10 vezes no erro em rampa como desejado.

\section*{12. Importância da Escolha Correta de \(a\) e \(b\)}

O par \(a\), \(b=10a\):

- determina quanto o lag desloca a magnitude em baixas frequências,
- deve ser suficientemente pequeno para não prejudicar a estabilidade,
- deve fornecer o ganho DC adequado para atingir o erro desejado.

% ============================================================
% ITEM ADICIONAL – INTEGRADO SEM ALTERAR A ESTRUTURA
% ============================================================

\section*{13. Código em Python para Ajuste Automático dos Parâmetros}

Além das análises teóricas e manuais, foi desenvolvido um código em Python para determinar automaticamente os melhores valores do ganho proporcional \(k\) e dos parâmetros \(a\) e \(b\) do compensador lag, visando atender simultaneamente às seguintes especificações:

\[
5\% \le M_p \le 15\%, \qquad
0.5 \le t_s \le 1.0 \text{ s}, \qquad
e_{ss} \le 1\% \text{ (degrau unitário)}.
\]

Durante as iterações, o algoritmo também buscou minimizar o ganho do compensador \(k_p\) e o erro em rampa, mantendo margens adequadas de estabilidade.

Os parâmetros ótimos encontrados foram:
\[
k = 76000, \qquad
a = 0.01, \qquad
b = 0.1.
\]

Com os seguintes resultados:
\[
\begin{aligned}
M_p &= 7.45\%, \\
t_s &= 0.531\text{ s},\\
e_{ss} &= 0.001\%,\\
k_v &= 95.93,\\
e_{\text{rampa}} &= 0.104\%.
\end{aligned}
\]

\begin{figure}
    \centering
    \includegraphics[width=0.8\linewidth]{Projeto_final_degral.png}
    \caption{Resposta ao Degral}
    \label{fig:placeholder}
\end{figure}

\begin{figure}
    \centering
    \includegraphics[width=0.8\linewidth]{Projeto_final_rampa.png}
    \caption{Resposta a Rampa}
    \label{fig:placeholder}
\end{figure}

\begin{figure}
    \centering
    \includegraphics[width=0.8\linewidth]{Projeto_final_bode.png}
    \caption{Diagrama de Bode}
    \label{fig:placeholder}
\end{figure}

\begin{figure}
    \centering
    \includegraphics[width=0.8\linewidth]{Projeto_final_nyquist.png}
    \caption{Diagrama de Nyquist}
    \label{fig:placeholder}
\end{figure}

Tais valores satisfazem as especificações e garantem estabilidade ao longo de toda a faixa de frequências do sistema.

% ============================================================
% CONCLUSÃO REFORMULADA PARA INCLUIR O ITEM 13
% ============================================================

\section*{Conclusão}

O estudo detalhado do ganho DC, da constante de velocidade e do comportamento do erro em rampa permitiu compreender plenamente o impacto do compensador lag no sistema. A análise teórica mostrou como a relação \(b/a=10\) aumenta \(k_v\) em dez vezes, reduzindo proporcionalmente o erro em rampa sem comprometer significativamente a estabilidade.

A integração com o código Python desenvolvido ampliou esse estudo ao permitir a busca automatizada de parâmetros ótimos. O conjunto final \(k=76000\), \(a=0.01\), \(b=0.1\) atendeu simultaneamente às especificações de sobresinal, tempo de acomodação e erros em regime permanente, garantindo desempenho robusto e estabilidade em toda a faixa de frequências.

Assim, o compensador lag projetado, aliado ao ganho proporcional adequado, fornece uma solução completa, precisa e otimizada, validada tanto analiticamente quanto computacionalmente.


\section*{Anexos}

\lstset{
    language=Python,
    basicstyle=\ttfamily\small,
    keywordstyle=\color{blue},
    stringstyle=\color{orange},
    commentstyle=\color{green!50!black},
    showstringspaces=false,
    frame=single,
    breaklines=true
}

Codigo para busca de parâmetros do compensador \(k_p\) e compensador lag.
\begin{lstlisting}
import numpy as np
import matplotlib.pyplot as plt
import control as ctl

# Planta do Projeto
km = 1.2
am = 13.2
ae = 950

s = ctl.TransferFunction.s
Gs = km/(s*(s+am)*(s+ae))

print("Planta G(s) =\n", )

def calcKp(funcGs, kinit, kend, inter):
    print(f"\nBuscando Kp entre {kinit} < Kp < {kend}, com {inter} posições")
    K_values = np.linspace(kinit, kend, inter)
    for ktest in K_values:
        print(".", end="")

        # sistema compensado sem realimentação: L(s) = K * G(s)
        Ls = ktest * funcGs

        # realimentação unitária
        Hf = ctl.feedback(Ls, 1)

        # resposta ao degrau
        t, y = ctl.step_response(Hf)

        # ---------- Especificações ----------
        # 1. Overshoot (em %)
        Mp = (max(y) - 1) * 100

        # 2. Tempo de acomodação 5%
        idx = np.where(abs(y - 1) > 0.05)[0]
        ts = t[idx[-1]] if len(idx) else 0

        # 3. Erro de regime para degrau
        ess = abs(1 - y[-1])

        # ---------- Filtros das especificações ----------
        if (
            5 <= Mp <= 15 and
            0.5 <= ts <= 1 and
            ess <= 0.01
        ):
            print(f"\n\nkp {ktest:.0f} atendeu TODAS as especificações.")
            return ktest
    print("\n\nNenhum kp atendeu TODAS as especificações.")


def calClag(funcGs, kp, Aint, Aend, inter):
    print(f"\nBuscando A e B do compensador entre {Aint} < Kp < {Aend}, com {inter} posições")
    resultados = []
    a_values = np.linspace(Aint, Aend, inter)

    for a in a_values:
        print(".", end="")
        b = 10 * a
        Clag = (s+b)/(s+a)

        # sistema compensado sem realimentação: L(s) = K * C(s) * G(s)
        Ls = kp * Clag * funcGs
        Hf = ctl.feedback(Ls, 1)  # realimentação unitária

        # resposta ao degrau
        t, y = ctl.step_response(Hf)

        # ---------- Especificações ----------
        # 1. Overshoot (em %)
        Mp = (max(y) - 1) * 100

        # 2. Tempo de acomodação 5%
        idx = np.where(abs(y - 1) > 0.05)[0]
        ts = t[idx[-1]] if len(idx) else 0

        # 3. Erro de regime para degrau
        ess = abs(1 - y[-1])

        # 4. Erro de rampa (Kv)
        Kv = kp * (km / (am + ae))
        er_rampa = 1 / Kv
        er_rampa_clag = 1 / (Kv * 10)

        # ---------- Filtros das especificações ----------
        if (
            5 <= Mp <= 15 and
            0.5 <= ts <= 1 and
            ess <= 0.01
        ):
            resultados.append(
                (a, b, kp, Mp, ts, ess, er_rampa, er_rampa_clag, Kv))

    return resultados


kresult = calcKp(Gs, 70000, 90000, 500)
SysOut = calClag(Gs, kresult, 0.01, 10, 200)

# Exibir resultados
if len(SysOut) == 0:
    print("\n\nNenhum conjunto (a, b, K) atendeu TODAS as especificações.")
else:
    print("\n\nSolução encontrada para o menor erro de rampa:")
    menor_Er_rampa = min(SysOut, key=lambda x: x[7])
    a, b, Kp, Mp, ts, ess, er_rampa, er_rampa_clag, Kv = menor_Er_rampa
    print(f"\nA = {a:.3f}, B = {b:.3f}, Kp = {Kp:.3f}")
    print(f"Overshoot Mp = {Mp:.2f}%")
    print(f"Tempo ts = {ts:.3f} s")
    print(f"Kv = {Kv:.3f}")
    print(f"Erro de regime (degrau) = {ess*100:.3f}%")
    print(f"Erro de rampa = {er_rampa*100:.3f}%")
    print(f"Erro de rampa Clag = {er_rampa_clag*100:.3f}%")

\end{lstlisting}

\vspace{1cm}

Código para geração de gráficos.
\begin{lstlisting}
import numpy as np
import matplotlib.pyplot as plt
import control as ctl

# Planta do Projeto
km = 1.2
am = 13.2
ae = 950

s = ctl.TransferFunction.s
Gs = km/(s*(s+am)*(s+ae))

# --- Compesador Proporcional
kcp = 77000

# --- Escolha a e b com b = 10*a para ganho DC = 10 ---
aLag = 0.01
bLag = 10.0*aLag
CLag = (s+bLag)/(s+aLag)
# CLag = 1

# --- Laço aberto e laço fechado (realiment. unitária com K=1) ---
Lk = kcp * Gs
Lkc = kcp * CLag * Gs
Gf = ctl.feedback(Gs, 1)
Gkf = ctl.feedback(Lk, 1)
Hf = ctl.feedback(Lkc, 1)

# resposta ao degrau
t, yt = ctl.step_response(Hf)

# ---------- Especificações ----------
# 1. Overshoot (em %)
Mp = (max(yt) - 1) * 100

# 2. Tempo de acomodação 5%
idx = np.where(abs(yt - 1) > 0.05)[0]
ts = t[idx[-1]] if len(idx) else 0

# 3. Erro de regime para degrau
ess = abs(1 - yt[-1])

# 4. Erro de rampa (Kv)
Kv = kcp * (km / (am + ae))
er_rampa = 1 / Kv
er_rampa_clag = 1 / (Kv * 10)

# --- Ganhos em DC ---
dcC = bLag/aLag
dcT = ctl.dcgain(Hf)

print("Ganho DC H(s) (fechado):", dcT)
print("C(s) = (s + {:.4g})/(s + {:.4g}) -> ganho DC do compensador = {:.3g}".format(bLag, aLag, dcC))
print(f"Overshoot Mp = {Mp:.2f}%")
print(f"Tempo ts = {ts:.3f} s")
print(f"Erro de regime (degrau) = {ess*100:.3f}%")
print(f"Kv = {Kv:.2f}")
print(f"Erro de rampa em k*G(s) = {er_rampa*100:.3f}%")
print(f"Erro de rampa em k*G(s)*C(s) = {er_rampa_clag*100:.3f}%")

# --- Bode ---
plt.figure(1)
Ws=np.logspace(-3,3,1000)
ctl.bode_plot(Gs, Ws, dB=True, label='G(s)')
ctl.bode_plot(Lk, Ws, dB=True, label='k*G(s)')
ctl.bode_plot(Lkc, Ws, dB=True, label='k*G(s)*C(s)')
plt.legend()

# --- Resposta ao degral ---
t1 = np.linspace(0, 3, 1000)
t1, y1 = ctl.step_response(Gf, t1)
t1, y2 = ctl.step_response(Gkf, t1)
t1, y3 = ctl.step_response(Hf, t1)
plt.figure(2)
plt.plot(t1, y1, label='G(s)')
plt.plot(t1, y2, label='k*G(s)')
plt.plot(t1, y3, label='k*G(s)*C(s)')
plt.title('Resposta ao degrau do sistema em malha fechada')
plt.xlabel("Tempo (s)")
plt.ylabel("Amplitude")
plt.legend()
plt.grid(True)

# --- Resposta a Rampa ---
t = np.linspace(0, 3, 1000)
rampa = t   # r(t) = t
t_out, y1 = ctl.forced_response(Gf, T=t, U=rampa)
t_out, y2 = ctl.forced_response(Gkf, T=t, U=rampa)
t_out, y3 = ctl.forced_response(Hf, T=t, U=rampa)

plt.figure(3)
plt.plot(t_out, y1, label="Saída G(s)")
plt.plot(t_out, y2, label="Saída k*G(s)")
plt.plot(t_out, y3, label="Saída k*G(s)*C(s)")
plt.plot(t_out, rampa, '--', label="Entrada rampa")
plt.title("Resposta à Rampa do sistema em malha fechada")
plt.xlabel("Tempo (s)")
plt.ylabel("Amplitude")
plt.grid(True)
plt.legend()

# --- Nyquist ---
plt.figure(4)
ctl.nyquist(Hf)
f_lim = 1 / (1 + 1j * np.array([0, 0, 1000, -1000]))
plt.plot(f_lim.real, f_lim.imag, 'mx')

plt.show()

\end{lstlisting}

\end{document}


